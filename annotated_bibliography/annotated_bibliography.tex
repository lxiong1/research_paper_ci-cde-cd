\documentclass[11pt,a4paper]{article}

\usepackage[utf8]{inputenc}
\usepackage[T1]{fontenc}
\usepackage[format=plain, font=it]{caption}
\usepackage{mathpazo}
\usepackage{changepage}
\usepackage[hyphens,spaces,obeyspaces]{url}

\begin{document}

\begin{titlepage} % Suppresses displaying the page number on the title page and the subsequent page counts as page 1
	\newcommand{\HRule}{\rule{\linewidth}{0.5mm}} % Defines a new command for horizontal lines, change thickness here
	\center
	\textsc{\LARGE University of Minnesota}\\[0.35cm]
	\textsc{\Large SENG 5852}\\[1.5cm]
	{\huge\bfseries Continuous Integration, Delivery, \& Deployment: Transforming the Software Industry}\\[0.35cm]	
	\textsc{\Large Annotated Bibliography}\\[1cm] 
	\textsc{\large\author
		LLue Xiong\\
	}
	\vfill\vfill\vfill
	\large\today
	\vfill
\end{titlepage}

\newpage
\section{Bibliography}
\nocite{*}
\bibliographystyle{annotate}
\bibliography{../bibliography}
%\begin{thebibliography}{6}
%	\bibitem{atkinson} 
%	Atkinson, B., \& Edwards, D. (2018). Generic Pipelines Using Docker: The 		DevOps Guide to Building Reusable, Platform Agnostic CI/CD Frameworks. Berkeley, CA: Apress. doi:\\\texttt{https://doi.org/10.1007/978-1-4842-3655-0}
%	
%	\bibitem{meyer}
%	Meyer, B. (2014). Agile! the Good, the Hype and the Ugly. Springer International Publishing. doi:\\\texttt{https://doi-org.ezp3.lib.umn.edu/10.1007/978-3-319-05155-0}
%	
%	\bibitem{bosch} 
%	Bosch, J. (2014). Continuous Software Engineering. Springer International Publishing. doi:\\\texttt{https://doi-org.ezp1.lib.umn.edu/10.1007/978-3-319-11283-1.}
%	\begin{adjustwidth}{1cm}{}
%		Jan discusses the unprecedented evolution of software engineering to respond to rapid
%		market changes and customer needs. The book and its studies emcompasses a group of
%		collaborators from the academic software researchers and software industries called
%		Software Center.
%		
%		Stairway to Heaven conceptual model is introduced as a way to understand the way
%		software industries transform to go against the competitive pressures of the market.
%		The transition from more traditional development practices to more agile-centric
%		development practices, resulting in continuous deployment of software to match the
%		pace of the ever changing market. When moving to agile practices, Jan says there must
%		be careful consideration of how it is introduced to an organization because it is a large
%		organizational mindset shift. It requires an organization to have smaller development
%		teams, cross-functional cooperation, and to think of software enhancements as pieces
%		of small modular change as opposed to a larger system change.
%		
%		Continuous integration is defined as having the ability to run automated test suite,
%		source controlled codebase as a means of continual delivery, and system architecture
%		that is modular in nature. However, continuous integration is found to be interpreted
%		and implemented differently from company to company and with varying degrees of
%		outcome. There are however similarities in all of the struggles for universal consensus of
%		what it means to be continuously integrated. Inherent in agile practices is the notion of
%		self-healing — using a collection of metrics such as static analysis, visual analytics and
%		business intelligence to combat software issues. These metrics are ran in an automated
%		environment, capturing data needed to take autonomous action for the purpose of
%		recovering from software abnormalities.
%		
%		As a result of the culture shift to an agile environment in the software industry,
%		traditional leadership roles have dramatically changed from a command and control
%		style to more engaging and supportive approach. Project managers, even if skilled, are
%		often left to find their own role in the fast-paced environment of agile culture, or face
%		leaving an organization altogether. The key to any transformation to a new culture Jan
%		says, is leadership culture change.
%	\end{adjustwidth}
%	
%	\bibitem{stackify} 
%	Continuous Delivery, Deployment \& Integration: 20 Key Differences. (2018, June 04). Retrieved from \\\texttt{https://stackify.com/continuous-delivery-vs-continuous-\\deployment-vs-continuous-integration}
%	
%	\bibitem{shahin}
%	 Shahin, M., Babar, M. A., \& Zhu, L. (2017). Continuous Integration, Delivery and Deployment: A Systematic Review on Approaches, Tools, Challenges and Practices. IEEE Access, 5, 3909-3943. doi:\\\texttt{10.1109/access.2017.2685629}
%	
%	\bibitem{groningen} 
%	St\aa hl, D. (2017). Large Scale Continuous Integration and Delivery: Making Great Software Better and Faster. [Groningen]: University of Groningen.
%	\begin{adjustwidth}{1cm}{}
%		Daniel Stahl’s doctoral thesis explores and investigates the changing culture of software
%		engineering in response to the increasingly software-dependent societies we live in. He
%		breaks up his thesis about continuous integration and delivery into three main parts.
%		The first part focuses on the nature of continuous integration, how it is interpreted, and
%		how it is implemented by different software entities. Daniel speaks to the fact that
%		continuous integration is rooted from a software development practice called eXtreme
%		programming — an agile practice created by Kent Beck. The idea of eXtreme
%		programming is to address software integrated changes early and often, of which
%		continuous integration as a part of the practice aims to aid in developer productivity,
%		agile testing, communication, and project predictability. Daniel states that the case
%		studies of continuous integration benefits by relative literature are merely suggestions,
%		rather than case studies backed with evidence. In an effort to provide evidence for
%		these claims, he presents evidence with a multi-case study of large-scale development
%		projects that have software professionals engaged in continuous integration as daily
%		practice. Explored also in the first part the doctoral thesis includes the discussion of
%		continuity and scalability of continuous integration. As project codebase grows larger,
%		test scope does as well; resulting in longer build compilation and test running time.
%		Daniel has found that there is a correlation between the size of a software organization
%		and its’ continuous integration tendencies.
%
%		The second part focuses on system modeling of continuous integration and delivery
%		solutions. Daniel attempts to close the gap of ambiguity for software professionals and
%		academic researchers in describing and designing continuous integration. Daniel
%		employs continuous integration modeling techniques in a multi-case study and analyzes
%		the effects of how software professionals understand their own continuous integration
%		systems.
%		
%		The third part focuses on the issue of traceability with continuous integration system
%		requirements. Traceability has long been known to the software industry as a practice
%		overwhelmed with its share of struggles. It is also a practice that is of considerable
%		importance to software engineering. In order to weave through the software industry
%		struggle, Daniel attempts to describe and evaluate possible solutions in context of
%		continuous integration systems.
%		
%		An observation of Daniel Stahl’s work on continuous integration and delivery seems to
%		suggests that he implicitly differentiates between continuous ‘integration’ and
%		‘delivery’. He often uses the term ‘continuous integration’ and has a seemingly
%		intentional habit to exclude the term “delivery”. It is a little concerning that he does not
%		explicitly speak to the differentiation though.
%	\end{adjustwidth}
%	
%	\bibitem{stahl} 
%	St\aa hl, D., Hall\'{e}n, K., \& Bosch, J. (2016). Achieving traceability in large scale continuous integration and delivery deployment, usage and validation of the eiffel framework. Empirical Software Engineering, 22(3), 967-995. doi:\\\texttt{10.1007/s10664-016-9457-1}
%	
%	
%\end{thebibliography}

\end{document}

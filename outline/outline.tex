\documentclass[11pt,a4paper]{article}

\usepackage[utf8]{inputenc}
\usepackage[T1]{fontenc}
\usepackage[format=plain, font=it]{caption}
\usepackage{mathpazo}
\usepackage{biblatex}

\begin{document}

\begin{titlepage} % Suppresses displaying the page number on the title page and the subsequent page counts as page 1
	\newcommand{\HRule}{\rule{\linewidth}{0.5mm}} % Defines a new command for horizontal lines, change thickness here
	\center
	\textsc{\LARGE University of Minnesota}\\[0.35cm]
	\textsc{\Large SENG 5852}\\[1.5cm]
	{\huge\bfseries Continuous Integration, Delivery, \& Deployment: Transforming the Software Industry}\\[0.35cm]	
	\textsc{\Large Outline}\\[1cm] 
	\textsc{\large\author
		LLue Xiong\\
	}
	\vfill\vfill\vfill
	\large\today
	\vfill
\end{titlepage}

\newpage

\section{Introduction}

\subsection{Thesis Statement}
The software industry is transforming at a rapid pace to accommodate the dynamic nature of the market and as a result, it continues to struggle to find process-identity with continuous software engineering.

\subsection{Purpose Statement}
Software engineering has for two decades, contemplated and experimented with the concept of distributing software in faster release cycles and have attempted to do so without sacrificing reliability and security. In attempt to achieve such a goal, there has been a widespread movement in the technical community to advocate for using Agile practices, and in particular: continuous integration, delivery, and deployment. The traditional methods of software development no longer meets the need of businesses that -- now more than ever -- want to proactively engage and retain their customers. The organizational transition to Agile practices demands a large mentality change, requiring individuals to recognize software as incremental features developed with cross-collaboration of small comprehensive team units as opposed to large modules developed by siloed units.

\section{Body}
	\subsection{What is Continuous Integration, Delivery, \& Deployment}
		\subsubsection{Inherently Agile}
		Explain what Agile is and how it ties in with CI/CDE/CD
		\subsubsection{Continuous Integration}
		Explain what CI is
		\subsubsection{Continuous Delivery}
		Explain what CDE is
		\subsubsection{Continuous Deployment}
		Explain what CD is
		
	\subsection{Differences of Interpretation \& Implementation}
		\subsubsection{Viewpoint of Software Professionals}
		Explain how software professionals interpret and implement CI/CD/CDE
		\subsubsection{Viewpoint of Academic Researchers}
		Explain how academic researchers interpret and how CI/CD/CDE should be implemented
		\subsubsection{Effort to Collaborate}
		Explain the uncollaborative phenomena of developers and researchers as well as the ongoing effort to bridge the gap
		
	\subsection{Benefits of Continuous Integration, Delivery, \& Deployment}
		\subsubsection{Self-healing Systems}
		Explain the metrics and tools that software professionals use to mitigate having to manually fix software issues
		\subsubsection{Reduce Risk}
		Explain how continuous software engineering reduces risk in systems 
		\subsubsection{Faster Release Cycles}
		Explain how faster release cycles are achieved
		\subsubsection{Overall Reduction of Cost}
		Explain why all of the above will reduce cost
		
	\subsection{Struggles of Traceability}
		\subsubsection{Importance}
		Explain the importance of traceability in the software engineering community
		\subsubsection{Problem of Mapping}
		Explain the problem of mapping requirements to implemented code and the converse
		\subsubsection{Eiffel Framework}
		Explain the proposed solution to address traceability issues in CI/CDE/CD environments
		
	\subsection{Transition an Agile Environment}
		\subsubsection{The Effect of Organizational Change to Agile}
		Explain the problems businesses face attempting to switch to Agile practices
		\subsubsection{General Roles}
		Explain typical roles that each individual plays in an Agile environment
		\subsubsection{Paradigm Shift in Leadership}
		Explain how leadership has changed as a result of Agile

\section{Conclusion}
	\subsection{Rephrase Thesis Statement}
	\subsection{Closing Statement}

\newpage
\section{Bibliography}
\begin{thebibliography}{6}
	\bibitem{Atkinson, Edwards} Atkinson, B., \& Edwards, D. (2018). Generic Pipelines Using Docker: The DevOps Guide to Building Reusable, Platform Agnostic CI/CD Frameworks. Berkeley, CA: Apress. doi:\\\texttt{https://doi.org/10.1007/978-1-4842-3655-0}
	\bibitem{Bosch} Bosch, J. (2014). Continuous Software Engineering. Cham: Springer International Publishing. doi:\\\texttt{https://doi-org.ezp1.lib.umn.edu/10.1007/978-3-319-11283-1.}
	\bibitem{Stackify} Continuous Delivery, Deployment \& Integration: 20 Key Differences. (2018, June 04). Retrieved from \\\texttt{https://stackify.com/continuous-delivery-vs-continuous-\\deployment-vs-continuous-integration}
	\bibitem{Shahin, Babar, Zhu} Shahin, M., Babar, M. A., \& Zhu, L. (2017). Continuous Integration, Delivery and Deployment: A Systematic Review on Approaches, Tools, Challenges and Practices. IEEE Access, 5, 3909-3943. doi:\\\texttt{10.1109/access.2017.2685629}
	\bibitem{Stahl} St\aa hl, D. (2017). Large Scale Continuous Integration and Delivery: Making Great Software Better and Faster. [Groningen]: University of Groningen.
	\bibitem{Stahl, Hallen, Bosch} St\aa hl, D., Hall\'{e}n, K., \& Bosch, J. (2016). Achieving traceability in large scale continuous integration and delivery deployment, usage and validation of the eiffel framework. Empirical Software Engineering, 22(3), 967-995. doi:\\\texttt{10.1007/s10664-016-9457-1}
\end{thebibliography}

\end{document}

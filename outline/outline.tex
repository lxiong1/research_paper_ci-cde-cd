\documentclass[11pt,a4paper]{article}

\usepackage[utf8]{inputenc}
\usepackage[T1]{fontenc}
\usepackage{enumitem}
\usepackage{mathpazo}
\usepackage[hyphens,spaces,obeyspaces]{url}
\usepackage{apacite}

\begin{document}

\begin{titlepage} % Suppresses displaying the page number on the title page and the subsequent page counts as page 1
	\newcommand{\HRule}{\rule{\linewidth}{0.5mm}} % Defines a new command for horizontal lines, change thickness here
	\center
	\textsc{\LARGE University of Minnesota}\\[0.35cm]
	\textsc{\Large SENG 5852}\\[1.5cm]
	{\huge\bfseries Continuous Integration, Delivery, \& Deployment: Transforming the Software Industry}\\[0.35cm]	
	\textsc{\Large Outline}\\[1cm] 
	\textsc{\large\author
		LLue Xiong\\
	}
	\vfill\vfill\vfill
	\large\today
	\vfill
\end{titlepage}

\newpage

\section{Introduction}

\subsection{Thesis Statement}
The software industry is transforming at a rapid pace to accommodate the dynamic nature of the market and as a result, it continues to struggle to find process-identity with continuous software engineering.

\subsection{Purpose Statement}
Software engineering has for two decades, experimented with the concept of distributing software in faster release cycles; endeavoring to do so without sacrificing reliability and security. To achieve such a goal, there has been a widespread movement in the technical community to advocate for using Agile practices, and in particular: continuous integration, delivery, and deployment. The traditional methods of software development no longer meet the need of businesses that -- now more than ever -- want to proactively engage and retain their customers. The organizational transition to Agile practices demands a large mentality change and require individuals to recognize software as incremental features developed with cross-collaboration of small comprehensive team units, as opposed to large modules developed by siloed units.

\section{Body}
	\subsection{What is Continuous Integration, Delivery, \& Deployment}
		\subsubsection{Inherently Agile}
		\begin{itemize}[noitemsep]
			\item What is Agile?
			\begin{itemize}
				\item
			\end{itemize}
			\item What are the core ideas of Agile?
			\begin{itemize}
				\item
			\end{itemize}
			\item How does Agile tie in with CI/CDE/CD?
			\begin{itemize}
				\item
			\end{itemize}
		\end{itemize}
		\vspace{-7mm}
		\subsubsection{Continuous Integration}
		\begin{itemize}[noitemsep]
			\item What does CI mean?
			\begin{itemize}
				\item
			\end{itemize}
		\end{itemize}
		\subsubsection{Continuous Delivery}
		\begin{itemize}[noitemsep]
			\item What does CDE mean?
			\begin{itemize}
				\item
			\end{itemize}
		\end{itemize}
		\subsubsection{Continuous Deployment}
		\begin{itemize}[noitemsep]
			\item What does CD mean?
			\begin{itemize}
				\item
			\end{itemize}
		\end{itemize}
		
	\subsection{Differences of Interpretation \& Implementation}
		\subsubsection{Viewpoint of Software Professionals}
		\begin{itemize}[noitemsep]
			\item How do software professionals interpret and implement CI/CD/CDE?
			\begin{itemize}
				\item
			\end{itemize}
		\end{itemize}
		\subsubsection{Viewpoint of Academic Researchers}
		\begin{itemize}[noitemsep]
			\item How do academic researchers interpret and believe how CI/CD/CDE should be implemented?
			\begin{itemize}
				\item
			\end{itemize}
		\end{itemize}
		\subsubsection{Collaboration Effort}
		\begin{itemize}[noitemsep]
			\item What effort is there to bridge the phenomena of non-collaboration between developers and researchers?
			\begin{itemize}
				\item
			\end{itemize}
		\end{itemize}

	\subsection{Benefits of Continuous Integration, Delivery, \& Deployment}
		\subsubsection{Self-healing Systems}
		\begin{itemize}[noitemsep]
			\item What are the metrics and tools that software professionals use to mitigate having to manually fix software issues?
			\begin{itemize}
				\item
			\end{itemize}
			\item How do these self-healing systems work?
			\begin{itemize}
				\item
			\end{itemize}
		\end{itemize}
		\subsubsection{Reduce Risk}
		\begin{itemize}[noitemsep]
			\item How does continuous software engineering reduces risk in systems?
			\begin{itemize}
				\item
			\end{itemize}
		\end{itemize}
		\subsubsection{Faster Release Cycles}
		\begin{itemize}[noitemsep]
			\item How are faster release cycles are achieved?
			\begin{itemize}
				\item
			\end{itemize}
		\end{itemize}
		\subsubsection{Overall Cost Reduction}
		\begin{itemize}[noitemsep]
			\item Why will all of the above will reduce cost?
			\begin{itemize}
				\item
			\end{itemize}
		\end{itemize}
		
	\subsection{Struggles of Traceability}
		\subsubsection{Importance}
		\begin{itemize}[noitemsep]
			\item What is the importance of traceability for the software engineering community?
			\begin{itemize}
				\item
			\end{itemize}
		\end{itemize}
		\subsubsection{Problem of Mapping}
		\begin{itemize}[noitemsep]
			\item What is the problem of mapping requirements to implemented code and the converse?
			\begin{itemize}
				\item
			\end{itemize}
		\end{itemize}
		\subsubsection{Eiffel Framework}
		\begin{itemize}[noitemsep]
			\item What is the proposed solution to address traceability issues in CI/CDE/CD environments?
			\begin{itemize}
				\item
			\end{itemize}
		\end{itemize}
		
	\subsection{Transition an Agile Environment}
		\subsubsection{The Effect of Organizational Change to Agile}
		\begin{itemize}[noitemsep]
			\item What are the problems that businesses face in attempt to switch to Agile practices?
			\begin{itemize}
				\item
			\end{itemize}
		\end{itemize}
		\subsubsection{Roles in Agile}
		\begin{itemize}[noitemsep]
			\item What are typical roles that each individual plays in an Agile environment?
			\begin{itemize}
				\item
			\end{itemize}
			\item Why do these roles exist?
			\begin{itemize}
				\item
			\end{itemize}
		\end{itemize}
		\subsubsection{Paradigm Shift in Leadership}
		\begin{itemize}[noitemsep]
			\item How has leadership changed as a result of Agile?
			\begin{itemize}
				\item
			\end{itemize}
		\end{itemize}

\section{Conclusion}
	\subsection{Rephrase Thesis Statement}
	\subsection{Closing Statement}

\newpage
\section{Bibliography}
\nocite{*}
\bibliographystyle{apacite}
\bibliography{../bibliography}

\end{document}
